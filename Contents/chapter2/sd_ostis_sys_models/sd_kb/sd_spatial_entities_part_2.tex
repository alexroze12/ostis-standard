\begin{SCn}
	
\scnheader{диаметр*}
\scniselement{бинарное отношение}
\scnrelfrom{первый домен}{пространственная сущность}
\scnrelfrom{второй домен}{отрезок}

\scnheader{высотная отметка*}
\scniselement{бинарное отношение}
\scnrelfrom{первый домен}{пространственная сущность}
\scnrelfrom{второй домен}{точка}

\scnheader{начальная точка*}
\scniselement{бинарное отношение}
\scnrelfrom{первый домен}{пространственная сущность}
\scnrelfrom{второй домен}{точка}

\scnheader{конечная точка*}
\scniselement{бинарное отношение}
\scnrelfrom{первый домен}{пространственная сущность}
\scnrelfrom{второй домен}{точка}

\scnheader{уровень}
\scniselement{измеряемый параметр}

\scnheader{абсцисса}
\scniselement{измеряемый параметр}

\scnheader{ордината}
\scniselement{измеряемый параметр}

\scnheader{прямоугольник}
\scnsubset{четырехугольник}
\scnexplanation{\textbf{\textit{Прямоугольник}} — \textit{четырёхугольник}, у которого все углы прямые (равны 90 градусам). Данная геометрическая фигура состоит из четырех \textit{точек}, которые соединены между собой двумя парами равных \textit{отрезков}, перпендикулярно пересекающихся только в этих \textit{точках}. Прямоугольник обладает следующими свойствами:
\begin{scnitemize}
	\item прямоугольник является параллелограммом — его противоположные стороны попарно параллельны;
	\item диагонали любого прямоугольника равны; 
	\item стороны прямоугольника являются его \textit{высотами}. Середины сторон прямоугольника образуют ромб;
	\item квадрат диагонали прямоугольника равен сумме квадратов двух его смежных сторон (по теореме Пифагора);
	\item около любого прямоугольника можно описать окружность, причём диагональ прямоугольника равна диаметру описанной окружности (радиус равен полудиагонали).
\end{scnitemize}
}

\scnheader{граничная точка*}
\scniselement{бинарное отношение}
\scnrelfrom{первый домен}{отрезок}
\scnrelfrom{второй домен}{точка}

\scnheader{вершина*}
\scniselement{бинарное отношение}
\scnrelfrom{второй домен}{точка}

\scnheader{отверстие*}
\scniselement{бинарное отношение}
\scnrelfrom{первый домен}{пространственная сущность}
\scnrelfrom{второй домен}{отверстие}

\scnheader{отверстие}
\scniselement{пространственная сущность}
\scnexplanation{\textbf{\textit{Отверстие}} - это полость в каком-либо предмете, обладающая ярковыраженными геометрическими свойствами. Обычно отверстие - это полость в виде цилиндра, сформированная (обычно) характеристиками вращающегося режущего объект инструмента.}

\scnheader{сечение*}
\scniselement{бинарное отношение}
\scnrelfrom{первый домен}{пространственная сущность}
\scnexplanation{Связки \textit{отношения} \textbf{\textit{сечение*}} связывают некоторую \textit{пространственную сущность} и замкнутую геометрическую фигуру, которая (1) лежит в некоторой \textit{плоскости}, имеющей общие точки с \textit{формой} данной \textit{пространственной сущности} и (2) граница которой включается в границу \textit{формы} данной \textit{пространственной сущности}.}

\scnheader{горизонтальное сечение*}
\scnsubset{сечение*}
\scnexplanation{Связки \textit{отношения} \textbf{\textit{горизонтальное сечение*}} связывают некоторую \textit{пространственную сущность} и ее \textit{сечение}, которое лежит в плоскости, перпендикулярной отрезку, являющемуся \textit{высотой} для данной пространственной сущности.}


\end{SCn}